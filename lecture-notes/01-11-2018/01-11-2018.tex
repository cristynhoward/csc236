\documentclass[12pt, oneside]{article} 
\usepackage[letterpaper, margin=0.5in]{geometry}        		
\geometry{letterpaper}
\usepackage[parfill]{parskip} 
\usepackage{framed}
\usepackage{graphicx}
\usepackage{amsmath}
\usepackage{amssymb}
\usepackage{qtree}
\usepackage{makecell}

\usepackage{mathtools}
\DeclarePairedDelimiter\ceil{\lceil}{\rceil}
\DeclarePairedDelimiter\floor{\lfloor}{\rfloor}
\DeclarePairedDelimiter\abs{\lvert}{\rvert}%
\DeclarePairedDelimiter\norm{\lVert}{\rVert}%
    % \abs & \norm resizes brackets, starred version doesn't
    \makeatletter
    \let\oldabs\abs
    \def\abs{\@ifstar{\oldabs}{\oldabs*}}
    %
    \let\oldnorm\norm
    \def\norm{\@ifstar{\oldnorm}{\oldnorm*}}
    \makeatother
    
\newcommand\tab[1][0.15cm]{\hspace*{#1}}
\newcommand\imp{\rightarrow}
\newcommand\thfr{\tab \therefore \tab}
\newcommand\sameas{\tab \equiv \tab}

\title{CSC236 - Week 2}
\author{Cristyn Howard}
\date{Thursday, January 11, 2018}

\begin{document}
\maketitle

\begin{itemize}
\item [Ex.] Prove that $\forall n \in \{\mathbb{N} \setminus\{ 0, 1, 2, 3, 4\}\}, \: n > 4, \: 2^n > n^2$

	\begin{itemize}
	\item Notation: Let $\mathbb{N}_{n>4} \equiv \{\mathbb{N} \setminus\{ 0, 1, 2, 3, 4\}\}$
	\item Let $P(n):  2^n>n^2.$
	
	\item Base case $n=5$: \tab $2^5 = 32 > 25 = 5^2 \tab \therefore P(5) = true.$
	\end{itemize} \vspace{0.7em}
	
\begin{leftbar}
\item [Note:] $\mathbb{N}_{n>4}$ is an infinite countable set. We can use a non-zero base case (e.g. 5) and use induction to prove that our predicate is true over an infinite countable subset of $\mathbb{N}$. \end{leftbar} \vspace{0.7em}

	\begin{itemize}
	\item Assume $p(k)$ holds for some arbitrary $k \in \mathbb{N}_{n>4} \thfr 2^k>k^2$.
		
		\begin{itemize}
		\item We have $2^k>k^2 \thfr 2\cdot 2^k>2k^2 \sameas$ \framebox{$2^{k+1} > 2k^2$}
		
		\item Note that $[[2^{k+1} > 2k^2] \land [2k^2 > (k+1)^2] \imp [P(k+1): 2^{k+1} > (k+1)^2]]$. So if we can show that $2k^2 > (k+1)^2$, we can conclude $P(k) \imp P(k+1)$.
		
		\item $2k^2 > (k+1)^2 \sameas 2k^2 > k^2 + 2k + 1 \sameas k^2 > 2k+1$ \newline
			$k^2 > 2k+1 \sameas k^2-2k-1>0 \sameas k^2-2k+1>2 \sameas (k-1)^2 > 2$
			
		\item $k \in \mathbb{N}_{n>4} \imp k > 4 \thfr (k-1)^2 > (4-1)^2 = 3^2 = 9 > 2 \thfr  (k-1)^2 > 2 = true$ 
		
		\item \framebox{$(k-1)^2 > 2 = true$}  $\imp$  
			\framebox{$ 2k^2 > (k+1)^2 = true$} $\imp$  
			\framebox{$ [P(k) \imp P(k+1)]= true$}
		\end{itemize}
		
		So we have shown that $P(k) \imp P(k+1)$.
		
	\item $P(5) \land [\: \forall k \in \mathbb{N}_{n>4}, \: P(k) \imp P(k+1)], \thfr$ by principle of simple induction, we can conclude that $\forall n \in\mathbb{N}_{n>4}, \: P(n)$.	
	\end{itemize}
	
\item [Ex.] Suppose that $h_0, h_1, h_2...$ is a sequence defined as follows: \newline
	\hspace{2em} $h_0 = 1$; \hspace{2em} $h_1 = 2$; \hspace{2em} $h_2 = 3$; \hspace{2em} 
	$h_k = h_{k-1} + h_{k-2} + h_{k-3} \:\:\: \forall k \in \mathbb{Z}, \: k \geq 3$. \newline
	Prove that $\forall n \in \mathbb{N}, h_n \leq 3^n$.
	
	\begin{itemize}
	\item Let $P(n):  h_n \leq 3^n.$
	
	\item Base case $n=0$: $h_n = 1 \leq 1 = 3^0 = 3^n$. \newline
	Base case $n=1$: $h_n = 2 \leq 3 = 3^1 = 3^n$. \newline
	Base case $n=2$: $h_n = 3 \leq 9 = 3^3 = 3^n$.
	
	\item Let k be some arbitrary $k \in \mathbb{N}$. Assume 
	\end{itemize}

\end{itemize}




\end{document}