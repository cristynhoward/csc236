\documentclass[12pt]{article} 
\usepackage[letterpaper, margin=0.5in]{geometry}        		
\geometry{letterpaper}
\usepackage[parfill]{parskip} 
\usepackage{framed}
\usepackage{graphicx}
\usepackage{amsmath}
\usepackage{amssymb}
\usepackage{qtree}
\usepackage{makecell}
\usepackage{lmodern}
\renewcommand*\familydefault{\sfdefault}
\usepackage{tikz}
\usetikzlibrary{matrix}

\usepackage{mathtools}
\DeclarePairedDelimiter\ceil{\lceil}{\rceil}
\DeclarePairedDelimiter\floor{\lfloor}{\rfloor}
\DeclarePairedDelimiter\abs{\lvert}{\rvert}%
\DeclarePairedDelimiter\norm{\lVert}{\rVert}%
    % \abs & \norm resizes brackets, starred version doesn't
    \makeatletter
    \let\oldabs\abs
    \def\abs{\@ifstar{\oldabs}{\oldabs*}}
    %
    \let\oldnorm\norm
    \def\norm{\@ifstar{\oldnorm}{\oldnorm*}}
    \makeatother

\newcommand\graytag[1]{\text{\textsl{\color{gray}{#1}}}}
\newcommand\tab[1][0.5cm]{\hspace*{#1}}
\newcommand\imp{\rightarrow}
\newcommand\thfr{\tab \therefore \tab}
\newcommand\sameas{\tab \equiv \tab}

\title{CSC236 - Week 3}
\author{Cristyn Howard}
\date{Thursday, January 18, 2018}

\begin{document}
\maketitle

\underline{Building Recursively Defined Sets}:
\begin{enumerate}
\item Define the smallest, simplest, elementary objects in the set.
\item Indicate how larger, more complex objects in the set can be constructed out of simpler ones.
\item Close the definition of the set.
\end{enumerate}
\vspace{0.5cm}

\underline{Examples of Recursively Defined Sets}:
\begin{itemize}
\item \textbf{The set of natural numbers, $\mathbb{N}$}.
	\begin{itemize}
	\item $0 \in \mathbb{N}$
	\item $k \in \mathbb{N} \rightarrow (k+1) \in \mathbb{N}$
	\item nothing else belongs to $\mathbb{N}$
	\end{itemize}

\item \textbf{Non-empty binary trees}.
	\begin{itemize}
	\item a single node is a binary tree
	\item given disjoint non-empty binary trees $T_1, \: T_2$, and single node $r$, \\ the tree with root $r$ connected to the roots of one or both of $\{T_1, \: T_2\}$ is a non-empty binary tree
	\item nothing else is a non-empty binary tree
	\end{itemize}
\end{itemize}
\vspace{0.5cm}

\underline{Structural Induction}: Prove that $P$ holds for all elements of a recursively defined set.
\begin{enumerate}
\item Show that every elementary object in the set satisfies $P$.
\item Assume that $P$ holds for smaller, simpler elements in the set. Show that every possible element constructed out of smaller elements for which $P$ holds also satisfies $P$.
\end{enumerate}
\vspace{0.5cm}

\underline{Examples of Structural Induction}: 
\begin{itemize}
\item Prove that every non-empty binary tree has one more node than edge.

\item Consider the following recursively defined set $S \subseteq \mathbb{N}^2$:
	\begin{description}
\item this is an item
\item this is an item
\item this is an item
\end{description}

\end{itemize}


\end{document}